% Options for packages loaded elsewhere
\PassOptionsToPackage{unicode}{hyperref}
\PassOptionsToPackage{hyphens}{url}
%
\documentclass[
]{article}
\usepackage{amsmath,amssymb}
\usepackage{iftex}
\ifPDFTeX
  \usepackage[T1]{fontenc}
  \usepackage[utf8]{inputenc}
  \usepackage{textcomp} % provide euro and other symbols
\else % if luatex or xetex
  \usepackage{unicode-math} % this also loads fontspec
  \defaultfontfeatures{Scale=MatchLowercase}
  \defaultfontfeatures[\rmfamily]{Ligatures=TeX,Scale=1}
\fi
\usepackage{lmodern}
\ifPDFTeX\else
  % xetex/luatex font selection
\fi
% Use upquote if available, for straight quotes in verbatim environments
\IfFileExists{upquote.sty}{\usepackage{upquote}}{}
\IfFileExists{microtype.sty}{% use microtype if available
  \usepackage[]{microtype}
  \UseMicrotypeSet[protrusion]{basicmath} % disable protrusion for tt fonts
}{}
\makeatletter
\@ifundefined{KOMAClassName}{% if non-KOMA class
  \IfFileExists{parskip.sty}{%
    \usepackage{parskip}
  }{% else
    \setlength{\parindent}{0pt}
    \setlength{\parskip}{6pt plus 2pt minus 1pt}}
}{% if KOMA class
  \KOMAoptions{parskip=half}}
\makeatother
\usepackage{xcolor}
\usepackage[margin = 0.75in]{geometry}
\usepackage{longtable,booktabs,array}
\usepackage{calc} % for calculating minipage widths
% Correct order of tables after \paragraph or \subparagraph
\usepackage{etoolbox}
\makeatletter
\patchcmd\longtable{\par}{\if@noskipsec\mbox{}\fi\par}{}{}
\makeatother
% Allow footnotes in longtable head/foot
\IfFileExists{footnotehyper.sty}{\usepackage{footnotehyper}}{\usepackage{footnote}}
\makesavenoteenv{longtable}
\usepackage{graphicx}
\makeatletter
\def\maxwidth{\ifdim\Gin@nat@width>\linewidth\linewidth\else\Gin@nat@width\fi}
\def\maxheight{\ifdim\Gin@nat@height>\textheight\textheight\else\Gin@nat@height\fi}
\makeatother
% Scale images if necessary, so that they will not overflow the page
% margins by default, and it is still possible to overwrite the defaults
% using explicit options in \includegraphics[width, height, ...]{}
\setkeys{Gin}{width=\maxwidth,height=\maxheight,keepaspectratio}
% Set default figure placement to htbp
\makeatletter
\def\fps@figure{htbp}
\makeatother
\setlength{\emergencystretch}{3em} % prevent overfull lines
\providecommand{\tightlist}{%
  \setlength{\itemsep}{0pt}\setlength{\parskip}{0pt}}
\setcounter{secnumdepth}{-\maxdimen} % remove section numbering
\ifLuaTeX
  \usepackage{selnolig}  % disable illegal ligatures
\fi
\IfFileExists{bookmark.sty}{\usepackage{bookmark}}{\usepackage{hyperref}}
\IfFileExists{xurl.sty}{\usepackage{xurl}}{} % add URL line breaks if available
\urlstyle{same}
\hypersetup{
  pdftitle={Oregon Birth Data},
  pdfauthor={Antonio Ceniceros},
  hidelinks,
  pdfcreator={LaTeX via pandoc}}

\title{Oregon Birth Data}
\author{Antonio Ceniceros}
\date{2025-10-30}

\begin{document}
\maketitle

\hypertarget{summary}{%
\section{Summary}\label{summary}}

\hypertarget{research-question-is-there-health-disparities-in-birth-outcomes-among-american-indianalaskan-native-populations-vs-other-racial-groups-in-the-state-of-oregon.}{%
\subsection{\texorpdfstring{\emph{Research question: Is there health
disparities in birth outcomes among American Indian/Alaskan Native
populations vs other racial groups in the state of
Oregon.}}{Research question: Is there health disparities in birth outcomes among American Indian/Alaskan Native populations vs other racial groups in the state of Oregon.}}\label{research-question-is-there-health-disparities-in-birth-outcomes-among-american-indianalaskan-native-populations-vs-other-racial-groups-in-the-state-of-oregon.}}

This analysis examines birth outcomes across different racial groups in
Oregon, focusing on low birth weight rates and prenatal care
utilization. Using population-level health data, we identified
significant disparities in birth outcomes and explored the relationship
between prenatal care timing and birth weight.

\hypertarget{key-findings}{%
\subsection{\texorpdfstring{\emph{Key
Findings}}{Key Findings}}\label{key-findings}}

\textbf{Summary stats}: People who identify as White are the majority of
total births (85\%) while American Indian/Alaskan Native accounted for
1.7\%

\begin{longtable}[]{@{}
  >{\raggedright\arraybackslash}p{(\columnwidth - 6\tabcolsep) * \real{0.5676}}
  >{\raggedleft\arraybackslash}p{(\columnwidth - 6\tabcolsep) * \real{0.1757}}
  >{\raggedleft\arraybackslash}p{(\columnwidth - 6\tabcolsep) * \real{0.1081}}
  >{\raggedleft\arraybackslash}p{(\columnwidth - 6\tabcolsep) * \real{0.1486}}@{}}
\caption{Birth Distribution by Race}\tabularnewline
\toprule\noalign{}
\begin{minipage}[b]{\linewidth}\raggedright
Race
\end{minipage} & \begin{minipage}[b]{\linewidth}\raggedleft
Total Births
\end{minipage} & \begin{minipage}[b]{\linewidth}\raggedleft
Records
\end{minipage} & \begin{minipage}[b]{\linewidth}\raggedleft
\% of Total
\end{minipage} \\
\midrule\noalign{}
\endfirsthead
\toprule\noalign{}
\begin{minipage}[b]{\linewidth}\raggedright
Race
\end{minipage} & \begin{minipage}[b]{\linewidth}\raggedleft
Total Births
\end{minipage} & \begin{minipage}[b]{\linewidth}\raggedleft
Records
\end{minipage} & \begin{minipage}[b]{\linewidth}\raggedleft
\% of Total
\end{minipage} \\
\midrule\noalign{}
\endhead
\bottomrule\noalign{}
\endlastfoot
White & 229474 & 918 & 84.9 \\
Asian & 14798 & 204 & 5.5 \\
More than one race & 12376 & 216 & 4.6 \\
Black or African American & 7545 & 204 & 2.8 \\
American Indian or Alaska Native & 4464 & 143 & 1.7 \\
Native Hawaiian or Other Pacific Islander & 1477 & 87 & 0.5 \\
\end{longtable}

\textbf{Statisical Analysis}: A chi-square test of independence revealed
a statistically significant association between race and low birth
weight status (p \textless{} 0.001), indicating that birth weight
outcomes vary significantly across racial groups.

\hypertarget{chi-square}{%
\subsubsection{\texorpdfstring{\emph{Chi
square}}{Chi square}}\label{chi-square}}

\begin{longtable}[]{@{}
  >{\raggedright\arraybackslash}p{(\columnwidth - 4\tabcolsep) * \real{0.5316}}
  >{\raggedleft\arraybackslash}p{(\columnwidth - 4\tabcolsep) * \real{0.2152}}
  >{\raggedleft\arraybackslash}p{(\columnwidth - 4\tabcolsep) * \real{0.2532}}@{}}
\caption{Contingency Table: Race by Low Birth Weight
Status}\tabularnewline
\toprule\noalign{}
\begin{minipage}[b]{\linewidth}\raggedright
\end{minipage} & \begin{minipage}[b]{\linewidth}\raggedleft
Low birth weight
\end{minipage} & \begin{minipage}[b]{\linewidth}\raggedleft
Normal birth weight
\end{minipage} \\
\midrule\noalign{}
\endfirsthead
\toprule\noalign{}
\begin{minipage}[b]{\linewidth}\raggedright
\end{minipage} & \begin{minipage}[b]{\linewidth}\raggedleft
Low birth weight
\end{minipage} & \begin{minipage}[b]{\linewidth}\raggedleft
Normal birth weight
\end{minipage} \\
\midrule\noalign{}
\endhead
\bottomrule\noalign{}
\endlastfoot
American Indian or Alaska Native & 78 & 4386 \\
Asian & 900 & 13898 \\
Black or African American & 430 & 7115 \\
More than one race & 564 & 11812 \\
Native Hawaiian or Other Pacific Islander & 0 & 1477 \\
White & 13751 & 215723 \\
\end{longtable}

\begin{longtable}[]{@{}lll@{}}
\caption{Chi-Square Test Results}\tabularnewline
\toprule\noalign{}
& Test Statistic & Value \\
\midrule\noalign{}
\endfirsthead
\toprule\noalign{}
& Test Statistic & Value \\
\midrule\noalign{}
\endhead
\bottomrule\noalign{}
\endlastfoot
X-squared & Chi-square & 276.84 \\
df & Degrees of Freedom & 5 \\
& P-value & \textless{} 0.001 \\
\end{longtable}

\hypertarget{analysis}{%
\subsubsection{\texorpdfstring{\emph{Analysis}}{Analysis}}\label{analysis}}

\begin{center}\includegraphics{births_data_files/figure-latex/analysis & plots-1} \end{center}

\begin{center}\includegraphics{births_data_files/figure-latex/analysis & plots-2} \end{center}

\hypertarget{limitations}{%
\subsection{Limitations}\label{limitations}}

This analysis has several important limitations:

\begin{enumerate}
\def\labelenumi{\arabic{enumi}.}
\item
  \textbf{Geographic Scope}: Data were originally requested for Klamath
  County; however, due to CDC WONDER's data suppression rules for small
  populations, this analysis utilized statewide Oregon data. This
  geographic aggregation may mask important county-level variations and
  limits the applicability of findings to Klamath County specifically.
\item
  \textbf{Ecological Fallacy}: Population-level associations cannot be
  used to infer individual-level relationships. Observed disparities do
  not account for individual confounders.
\item
  \textbf{Unmeasured Confounding}: Important factors such as maternal
  education, income, insurance status, healthcare access, maternal age,
  and pre-existing conditions are not included in this analysis.
\item
  \textbf{Data Suppression}: CDC WONDER suppresses data for privacy
  protection, potentially resulting in incomplete representation of
  smaller racial/ethnic populations, particularly affecting American
  Indian/Alaska Native populations.
\item
  \textbf{Prenatal Care Measurement}: Analysis captures only timing of
  care initiation, not quality, frequency, or content of prenatal
  services received.
\item
  \textbf{Race/Ethnicity Classification}: Self-reported categories may
  not capture the full complexity of identity and within-group
  heterogeneity. For Native American populations specifically, this
  classification does not account for tribal affiliation, urban versus
  reservation residence, or degree of connection to traditional
  healthcare systems.
\end{enumerate}

\hypertarget{conclusions}{%
\subsection{Conclusions}\label{conclusions}}

This statewide analysis of Oregon birth outcomes reveals significant
racial disparities in birth outcomes. The findings suggest that:

\begin{enumerate}
\def\labelenumi{\arabic{enumi}.}
\tightlist
\item
  Low birth weight rates vary significantly across racial groups (p
  \textless{} 0.001)
\item
  Birth weight distributions differ substantially across populations
\end{enumerate}

While statewide data provide important insights into population-level
health disparities, the inability to access county-specific data limits
conclusions about Klamath County specifically. The geographic
aggregation is particularly problematic for understanding American
Indian/Alaska Native birth outcomes, as these populations may have
unique health service utilization patterns and risk factors that vary by
tribal affiliation and geographic location.

\textbf{Recommendations for Future Research:}

To better understand birth outcome disparities, particularly for Native
American populations, future studies should: (1) partner with Tribal
Health Organizations and Tribal Epidemiology Centers for more detailed
community-specific data; (2) incorporate cultural factors and assess
specific healthcare access barriers including geographic isolation,
transportation, and culturally competent care availability; (3) examine
differences between urban and reservation populations; and (4) employ
Community-Based Participatory Research approaches that center Native
voices and ensure findings benefit the communities studied. These
approaches would provide more nuanced understanding and inform targeted,
culturally appropriate interventions.

\begin{center}\rule{0.5\linewidth}{0.5pt}\end{center}

\textbf{Methods}: Data were analyzed using R (version 4.2.1).
Statistical significance was assessed using chi-square tests. All
visualizations created using ggplot2.

\textbf{Code}: Full code can be found at this link:
\url{https://github.com/tonioceniceros/birth_data/blob/main/births_data.Rmd}

\end{document}
